%!TEX TS-program = xelatex
\documentclass[]{friggeri-cv}
\usepackage{afterpage}
\usepackage{hyperref}
\usepackage{color}
\usepackage{xcolor}
\usepackage{smartdiagram}
\usepackage{fontspec}
% if you want to add fontawesome package
% you need to compile the tex file with LuaLaTeX
% References:
%   http://texdoc.net/texmf-dist/doc/latex/fontawesome/fontawesome.pdf
%   https://www.ctan.org/tex-archive/fonts/fontawesome?lang=en
%\usepackage{fontawesome}
\usepackage{metalogo}
\usepackage{dtklogos}
\usepackage[utf8]{inputenc}
\usepackage{tikz}
\usepackage{enumitem}
\usetikzlibrary{mindmap,shadows}
\hypersetup{
    pdftitle={Florent Clarret - CV},
    pdfauthor={Florent Clarret},
    pdfsubject={CV},
    pdfkeywords={Clarret, resume, cv},
    colorlinks=false,           % no lik border color
    allbordercolors=white       % white border color for all
}
\smartdiagramset{
    bubble center node font = \footnotesize,
    bubble node font = \footnotesize,
    % specifies the minimum size of the bubble center node
    bubble center node size = 0.5cm,
    %  specifies the minimum size of the bubbles
    bubble node size = 0.5cm,
    % specifies which is the distance among the bubble center node and the other bubbles
    distance center/other bubbles = 0.3cm,
    % sets the distance from the text to the border of the bubble center node
    distance text center bubble = 0.5cm,
    % set center bubble color
    bubble center node color = pblue,
    % define the list of colors usable in the diagram
    set color list = {lightgray, materialcyan, orange, green, materialorange, materialteal, materialamber, materialindigo, materialgreen, materiallime},
    % sets the opacity at which the bubbles are shown
    bubble fill opacity = 0.6,
    % sets the opacity at which the bubble text is shown
    bubble text opacity = 0.5,
}

\addbibresource{bibliography.bib}
\RequirePackage{xcolor}
\definecolor{pblue}{HTML}{0395DE}

\begin{document}

\header{Florent}
				{Clarret} 
				{Ingénieur étude et développement}
				{+33 (0)6 62 97 01 69}
				{\href{mailto:florent.clarret@gmail.com}{\textbf{florent.clarret@}gmail.com} }
				{\href{https://www.linkedin.com/in/florent-clarret/}{florent-clarret} \includegraphics{img/linkedin.png} }
				{\href{https://github.com/FlorentClarret}{FlorentClarret} \includegraphics{img/github.png}}
	
% Fake text to add separator      
\fcolorbox{white}{gray}{\parbox{\dimexpr\textwidth-2\fboxsep-2\fboxrule}{%
.....
}}


% In the aside, each new line forces a line break
\begin{aside}
  \includegraphics[scale=0.2]{img/photo.png} 
    \section{Infos}
    Né le 01/04/1992
    Nationalité française
    Permis B et véhicule
    ~
  \section{Langues}
    Anglais (TOEIC 960)
    ~
  \section{Développement}
    \smartdiagram[bubble diagram]{
        \textbf{Java},
        \textbf{Shell},
        \textbf{Spring}\\\textbf{Boot},
        \textbf{JPA},
        \textbf{Tomcat},
        \textbf{Active}\\\textbf{MQ},
        \textbf{Linux},
        \textbf{Oracle},
        \textbf{MySQL}
    }
    ~
  \section{Outillage}
    \smartdiagram[bubble diagram]{
        \textbf{Tools},
        \textbf{Maven},
        \textbf{Gitlab},
        \textbf{Docker},
        \textbf{CI/CD},
        \textbf{Git},
        \textbf{Dynatrace},
        \textbf{Jmeter},
        \textbf{JMH}
    }
    ~
   \section{Side projects}
    \textit{\#Homelab}
    \textit{\#DIY}
    \textit{\#HomeServer}
    \textit{\#HomeAutomation}
    ~
\end{aside}
~
\\
\\
\section{Expériences professionnelles}
\begin{entrylist}
      \vspace{7pt}    
  \entry
    {Depuis 2018}
    {Ingénieur étude et développement}
    {Worldline, Tours}
    {
    \vspace{-0.8\baselineskip}
	\begin{itemize}[leftmargin=*]
		\item Refonte technique et évolution fonctionnelle du Back-End d'une solution d'authentificiation forte des personnes.
		\item Analyse des besoins des utilisateurs.
		\item Participation au choix des solutions techniques pour la mise en place de nouvelles fonctionnalités.
		\item Ajout de moyens alternatifs d'authentification par fingerprint ou par le protocole FIDO en plus de l'utilisation du PIN. 
		\item Conception et développement d'une nouvelle API permettant une communication client/serveur plus efficace et sécurisée.
		\item Mise en place de procédés cryptographiques pour les échanges client/serveur.
		\item Dockerisation de l'application pour faciliter la mise en place des tests d'intégration.
		\item Support actif aux équipes chargées de la production.
		\item Java avec Spring 3 déployée sur Tomcat. Migration vers Spring Boot 2. MySQL.
	\end{itemize}
	}
      \vspace{7pt}    
  \entry
    {Depuis 2017}
    {Chargé d'enseignement - 64h par an}
    {Polytech Tours}
    {
    \vspace{-0.8\baselineskip}
	\begin{itemize}[leftmargin=*]
		\item Cours pour les étudiants de dernière année du cursus d'ingénieur.
		\item Développement en Java, son écosystème ainsi que sur l'aspect performance.
		\item Monitoring d'applications, mise en place de benchmarks, utilisation de JMeter, microbenchmarks JMH, étude et amélioration d'applications existantes.
	\end{itemize}
	}
      \vspace{7pt}    
    \entry
    {2015 - 2018}
    {Ingénieur étude et développement}
    {Worldline, Tours}
    {
    \vspace{-0.8\baselineskip}
    \begin{itemize}[leftmargin=*]
		\item Etude, développement, optimisation et déploiement en production de la nouvelle version du Back-End de la solution de paiement sur internet SIPS.
		\item Développement principalement Java, déployé en standalone ou sur des Tomcat ou JBoss,  avec une base de données Oracle.
		\item Mise  en place des nouveaux traitements pour assurer la sauvegarde et le traitement des transactions bancaires.
		\item Participation active à la migration en production des clients sur cette nouvelle solution et mise en place de surveillances.
		\item Maintenance de la première version de la solution lors de la phase de transition (Base de données Sybase \& MySQL, programme C, scripts Shell \& Awk).
		\item Suivi et support aux équipes situées à l'étranger (Inde et Arménie).
	\end{itemize}
}
      \vspace{7pt}      
    \entry
    {2013 - 2015}
    {Etudiant-enseignant - 2h par semaine}
    {Polytech Tours}
    {
    \vspace{-0.8\baselineskip}
	\begin{itemize}[leftmargin=*]
		\item Cours personnalisés aux étudiants de première année de cycle d'ingénieur. 
		\item Algorithmique, structures de données et implémentation en C.
	\end{itemize}
    }
\end{entrylist}

\section{Formation}
\begin{entrylist}
  \vspace{-6pt}    
  \entry
    {2012 - 2015}
    {Diplôme d'ingénieur en informatique}
    {Polytech Tours}
    {}
  \vspace{-6pt}    
  \entry
    {2014 - 2014}
    {Diplôme d'ingénieur en informatique}
    {Politechnika Łódzka, Pologne}
    {}
  \entry
    {2010 - 2012}
    {Diplôme universitaire de technologie en informatique}
    {IUT de Lens}
    {}
\end{entrylist}

\section{Centres d'intérêts}
\begin{entrylist}
  \entry
    {Homelaber - Lecture (Fantasy \& mangas) - Course à pied \& Canicross}
    {}    {}    {}
\end{entrylist}

\end{document}
